\documentclass[zihao=-4,fontset=none]{ctexart}
\usepackage{array,enumitem,geometry,makecell,xpinyin,hyperref}

\setmainfont{FreeSerif}
\setsansfont{FreeSans}
\setCJKmainfont{FZShuSong-Z01}[ItalicFont=FZKai-Z03]
\setCJKsansfont{FZHei-B01}
\newCJKfontfamily\kaishu{FZKai-Z03}

\pagestyle{plain}
\ctexset{
  punct=kaiming,
  section/number=\chinese{section},
  section/format=\large\sffamily\centering}
\geometry{
  a4paper,
  vmargin=2.54cm,
  hmargin=3.18cm,
  headheight=15pt}
\setlist[enumerate]{
  label=(\arabic*),
  parsep=0pt,
  itemsep=0pt,
  topsep=0pt}

\newcolumntype{V}{!{\vrule width 1pt}}
\newcolumntype{C}[1]{>{\centering\arraybackslash}m{#1}}

% ㄪ (BOPOMOFO LETTER V): U+312A
% 万 (wan): U+4E07
\def\bopomofov{\textit{\symbol{"4E07}}}

\hypersetup{
  bookmarksnumbered=true,
  pdftitle={汉语拼音方案}}

\title{\huge\sffamily
  \begin{CJKfilltwosides}{8em}
    汉语拼音方案
  \end{CJKfilltwosides}}
\author{\normalsize\kaishu
  (1958 年 2 月 11 日第一届全国人民代表大会第五次会议通过)}
\date{}

\begin{document}

\maketitle

\section{字母表}

\begin{center}
  \begin{tabular}{*{8}{c}}
    \textsf{字母:} & A a  & B b  & C c    & D d  & E e  & F f  & G g  \\
    \textsf{名称:} & ㄚ   & ㄅㄝ & ㄘㄝ   & ㄉㄝ & ㄜ   & ㄝㄈ & ㄍㄝ \\
    \cline{2-8}
                    & H h  & I i  & J j    & K k  & L l  & M m  & N n  \\
                    & ㄏㄚ & ㄧ   & ㄐㄧㄝ & ㄎㄝ & ㄝㄌ & ㄝㄇ & ㄋㄝ \\
    \cline{2-8}
                    & O o  & P p  & Q q    & R r  & S s  & T t  & \\
                    & ㄛ   & ㄆㄝ & ㄑㄧㄡ & ㄚㄦ & ㄝㄙ & ㄊㄝ & \\
    \cline{2-7}
                    & U u  & V v  & W w    & X x  & Y y  & Z z  & \\
                    & ㄨ   & \bopomofov ㄝ & ㄨㄚ   & ㄒㄧ & ㄧㄚ & ㄗㄝ &
  \end{tabular}
\end{center}

V 只用来拼写外来语、少数民族语言和方言。

字母的手写体依照拉丁字母的一般书写习惯。

\section{声母表}

\begin{center}
  \begin{tabular}{*{9}{c}}
    b    & p    & m    & f    &  & d    & t    & n    & l    \\
    ㄅ玻 & ㄆ坡 & ㄇ摸 & ㄈ佛 &  & ㄉ得 & ㄊ特 & ㄋ讷 & ㄌ勒 \\
    \hline
    g    & k    & h    &      &  & j    & q    & x    &      \\
    ㄍ哥 & ㄎ科 & ㄏ喝 &      &  & ㄐ基 & ㄑ欺 & ㄒ希 &      \\
    \hline
    zh   & ch   & sh   & r    &  & z    & c    & s    &      \\
    ㄓ知 & ㄔ蚩 & ㄕ诗 & ㄖ日 &  & ㄗ资 & ㄘ雌 & ㄙ思 &
  \end{tabular}
\end{center}

在给汉字注音的时候,为了使拼式简短,zh ch sh 可以省作 ẑ ĉ ŝ。

\section{韵母表}

\begin{center}
  \begin{tabular}{Vp{4cm}|p{2.5cm}|p{2.5cm}|p{2.5cm}V}
    \Xcline{1-4}{1pt}
        & i    & u    & ü   \\                  & ㄧ   ~ 衣 & ㄨ   ~ 乌 & ㄩ   ~ 迂 \\ \hline
    a   & ia   & ua   &     \\ ㄚ ~ 啊          & ㄧㄚ ~ 呀 & ㄨㄚ ~ 蛙 &           \\ \hline
    o   &      & uo   &     \\ ㄛ ~ 喔          &           & ㄨㄛ ~ 窝 &           \\ \hline
    e   & ie   &      & üe  \\ ㄜ ~ 鹅          & ㄧㄝ ~ 耶 &           & ㄩㄝ ~ 约 \\ \hline
    ai  &      & uai  &     \\ ㄞ ~ 哀          &           & ㄨㄞ ~ 歪 &           \\ \hline
    ei  &      & uei  &     \\ ㄟ ~ 欸          &           & ㄨㄟ ~ 威 &           \\ \hline
    ao  & iao  &      &     \\ ㄠ ~ 熬          & ㄧㄠ ~ 腰 &           &           \\ \hline
    ou  & iou  &      &     \\ ㄡ ~ 欧          & ㄧㄡ ~ 忧 &           &           \\ \hline
    an  & ian  & uan  & üan \\ ㄢ ~ 安          & ㄧㄢ ~ 烟 & ㄨㄢ ~ 弯 & ㄩㄢ ~ 冤 \\ \hline
    en  & in   & uen  & ün  \\ ㄣ ~ 恩          & ㄧㄣ ~ 因 & ㄨㄣ ~ 温 & ㄩㄣ ~ 晕 \\ \hline
    ang & iang & uang &     \\ ㄤ ~ 昂          & ㄧㄤ ~ 央 & ㄨㄤ ~ 汪 &           \\ \hline
    eng & ing  & ueng &     \\ ㄥ ~ 亨的韵母    & ㄧㄥ ~ 英 & ㄨㄥ ~ 翁 &           \\ \hline
    ong & iong &      &     \\ (ㄨㄥ)轰的韵母 & ㄩㄥ ~ 雍 &           &           \\
    \Xcline{1-4}{1pt}
  \end{tabular}
\end{center}

\begin{enumerate}
  \item “知、蚩、诗、日、资、雌、思”等七个音节的韵母用 i,即:知、蚩、诗、日、资、雌、思等字拼作
        zhi,chi,shi,ri,zi,ci,si。

  \item 韵母ㄦ写成 er,用作韵尾的时候写成 r。例如:“儿童”拼作 ertong,“花儿” 拼作 huar。

  \item 韵母ㄝ单用的时候写成 ê。

  \item i 行的韵母,前面没有声母的时候,写成 yi(衣),ya(呀),ye(耶),yao(腰),you(忧),
        yan(烟),yin(因),yang(央),ying(英),yong(雍)。

    u 行的韵母,前面没有声母的时候,写成 wu(乌),wa(蛙),wo(窝),wai(歪),wei(威),
        wan(弯),wen(温),wang(汪),weng(翁)。

    ü 行的韵母,前面没有声母的时候,写成 yu(迂),yue(约),yuan(冤),yun(晕);ü 上两点省略。

    ü 行的韵母跟声母 j,q,x 拼的时候,写成 ju(居),qu(区),xu(虚),ü 上两点也省略;但是跟
        声母 n,l 拼的时候,仍然写成 nü(女),lü(吕)。

  \item iou,uei,uen 前面加声母的时候,写成 iu,ui,un。例如 niu(牛),ɡui(归),lun(论)。

  \item 给汉字注音的时候,为了使拼式简短,nɡ 可以省作 ŋ。
\end{enumerate}

\section{声调符号}

\begin{center}
  \begin{tabular}{*{4}{C{3em}}}
    阴平 & 阳平 & 上声 & 去声 \\
    ˉ    & ˊ    & ˇ    & ˋ
  \end{tabular}
\end{center}

声调符号标在音节的主要母音上。轻声不标。例如:

\begin{center}
  \begin{tabular}{*{5}{C{4em}}}
    妈 \pinyin{ma1} & 麻 \pinyin{ma2} & 马 \pinyin{ma3} & 骂 \pinyin{ma4} & 吗 \pinyin{ma} \\
    (阴平)        & (阳平)        & (上声)        & (去声)        & (轻声)
  \end{tabular}
\end{center}

\section{隔音符号}

a,o,e 开头的音节连接在其它音节后面的时候,如果音节的界限发生混淆,用隔音符号(')隔开,例如:
\pinyin{pi'ao}(皮袄)。

\end{document}
